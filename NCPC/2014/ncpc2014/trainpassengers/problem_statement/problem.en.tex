\problemname{Train Passengers}

\illustration{.2}{Train_stuck_in_snow}{Photo by
  \href{https://commons.wikimedia.org/wiki/File:Train_stuck_in_snow.jpg}{Elmer and Tenney}}

The Nordic Company of Passing Carriages is losing money at an alarming
rate because most of their trains are empty. However, on some lines
the passengers are complaining that they cannot fit in the cars and
have to wait for the next train!

The authorities want to fix this situation. They asked their station
masters to write down, for a given train, how many people left the
train at their station, how many went in, and how many had to
wait. Then they hired your company of highly paid consultants to
assign properly sized trains to their routes.

You just received the measurements for a train, but before feeding
them to your optimisation algorithm you remembered that they were
collected on a snowy day, so any sensible station master would have
preferred to stay inside their cabin and make up the numbers instead
of going outside and counting.

Verify your hunch by checking whether the input is inconsistent, i.e.,
at every time the number of people in the train did not exceed the
capacity nor was below 0 and no passenger waited in vain. The train
should start and finish the journey empty, in particular passengers
should not wait for the train at the last station. At each station passengers first leave, then
others enter.

\section*{Input}

The first line contains two integers $C$ and $n$ ($1 \leq C \leq 10^9$, $2 \leq n \leq
100$), the total capacity and the number of stations the train stops
in. The next $n$ lines contain three integers each, the number of
people that left the train, entered the train, and had to stay at a
station. Lines are given in the same order as the train visits each
station. All integers are between $0$ and $10^9$ inclusive.

\section*{Output}

One line containing one word: \verb+possible+ if the measurements are
consistent, \verb+impossible+ otherwise.
