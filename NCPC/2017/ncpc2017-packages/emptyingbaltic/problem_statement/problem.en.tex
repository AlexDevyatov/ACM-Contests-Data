\problemname{Emptying the Baltic}

\illustration{0.4}{floodbank}{\href{https://www.flickr.com/photos/anax/8038782084/}{Picture} by Jeremy Halls on Flickr, cc by-sa}%
Gunnar dislikes forces of nature and always comes up with innovative plans to decrease their
influence over him. Even though his previous plan of a giant dome over Stockholm to protect from too
much sunlight (as well as rain and snow) has not yet been realized, he is now focusing on preempting
the possible effects climate change might have on the Baltic Sea, by the elegant solution of simply
removing the Baltic from the equation.

First, Gunnar wants to build a floodbank connecting Denmark and Norway to separate the Baltic from
the Atlantic Ocean. The floodbank will also help protect Nordic countries from rising sea levels in the
ocean. Next, Gunnar installs a device that can drain the Baltic from the seafloor. The device will
drain as much water as needed to the Earth's core where it will disappear forever (because that is how physics works, at least as far as Gunnar is concerned).  However, depending on the placement of the device, the entire Baltic might not be completely drained -- some pockets of water may remain.

To simplify the problem, Gunnar is approximating the map of the Baltic using a $2$-dimensional grid
with $1$ meter squares. For each square on the grid, he computes the average altitude. Squares
with negative altitude are covered by water, squares with non-negative altitude are dry. Altitude is
given in meters above the sea level, so the sea level has altitude of exactly $0$. He
disregards lakes and dry land below the sea level, as these would not change the estimate much
anyway.

Water from a square on the grid can flow to any of its $8$~neighbours, even if the two squares only
share a corner. The map is surrounded by dry land, so water never flows outside of the map. Water
respects gravity, so it can only flow closer to the Earth's core -- either via the drainage device
or to a neighbouring square with a lower water level.

Gunnar is more of an idea person than a programmer, so he has asked for your help to evaluate how much water would be drained for a given placement of the device.

% TODO: statement suggests that the water body is connected. Do we want to guarantee that?


\section*{Input}

The first line contains two integers $h$ and $w$, $1 \leq h, w \leq 500$, denoting the height and
width of the map.

Then follow $h$ lines, each containing $w$ integers. The first line represents the northernmost row
of Gunnar's map. Each integer represents the altitude of a square on the map grid. The altitude is
given in meters and it is at least $-10^6$ and at most $10^6$.

The last line contains two integers $i$ and $j$, $1 \leq i \leq h,
1 \leq j \leq w$, indicating that the draining device is placed in the
cell corresponding to the $j$'th column of the $i$'th row.  You may
assume that position $(i, j)$ has negative altitude (i.e., the
draining device is not placed on land).


\section*{Output}

Output one line with one integer -- the total volume of sea water drained, in cubic meters.
