\problemname{Floppy Music}
\illustration{.4}{floppy2}{Antoine Taveneaux, cc-by-sa}

Your friend's newest hobby is to play movie theme songs on her freshly
acquired floppy drive organ. This organ is a collection of good old
floppy drives, where each drive has been tampered with to produce sound
of a unique frequency. The sound is produced by a step motor that moves
the read/write head of the floppy drive along the radial axis of the drive’s spin disk. The radial
axis starts in the center of the spin disk and ends at the outer edge of the spin disk.

The sound from one drive will
play continuously as long as the read/write head keeps moving in one
direction; when the head changes direction, there is a brief pause of 
$1$fs---one floppysecond, or about $100$ microseconds. The
read/write head must change direction when it reaches either the inner
or the outer end point of the radial axis, but it can also change
direction at any other point along this axis, as determined by your
friend. You can make the head stay still at any time and for as long as you wish.
The starting position of the read-write head can be chosen freely.

Your friend is a nutcase perfectionist, and will not accept any pauses
where there are not supposed to be any; nor will she accept sound when
there is meant to be silence. To figure out whether a given piece of
music can be played---perfectly---on her organ, she has asked for
your help.

For each frequency, you are given a list of intervals, each
describing when that particular frequency should play, and you must decide
if all of the frequencies can be played as intended. You can assume your
friend has enough drives to cover all the required frequencies.

\section*{Input}
The first line contains an integer~$f, 1 \leq f \leq 10$, denoting the number of frequencies used.
Then follow~$f$ blocks, on the format:
\begin{itemize}
  \item A single line with two integers~$t_i, 1\leq t_i \leq 10\ 000$ and $n_i, 1\leq n_i \leq 100$; the number of floppyseconds
  it takes for the read/write head of frequency~$i$ to move between the end points of its radial axis, and the number
  of intervals for which frequency~$i$ should play.
  \item $n_i$ lines, where the~$j$-th line has two integers~$t_{i,2j}, t_{i,2j+1}$, where $0\leq t_{i,2j},t_{i,2j+1} \leq 1\ 000\ 000$, indicating that the~$i$-th frequency should
  start playing at time $t_{i,2j}$ and stop playing at time $t_{i,2j+1}$. You can assume that these numbers are in stricly ascending order, i.e. 
  $t_{i,1} < t_{i,2} < \dots < t_{i, 2n_i}$. 

% \item an integer~$t_i, 1\leq t_i \leq 10\ 000$, which is the number of floppyseconds it takes for the read/write head of
% frequency~$i$ to move between the end points of its radial axis.
% \item an integer~$n_i, 1\leq n_i \leq 100$, denoting the number of intervals
% for which frequency~$i$ should play.
% \item $2n_i$ integers~$t_{i,j}, 0\leq t_{i,j} \leq 1\ 000\ 000$ where each $t_{i,j}$ denotes a toggle
% (on/off) for frequency~$i$ at time $t_{i,j}$. You can assume that these numbers are in strictly
% ascending order, i.e. $t_{i,1} < t_{i,2} < \dots < t_{i, 2n_i}$.
\end{itemize}
% All frequencies are initially switched off.

\section*{Output}
If it is possible to play all the $f$ frequencies as intended, output
``\texttt{possible}''. Otherwise output ``\texttt{impossible}''.

