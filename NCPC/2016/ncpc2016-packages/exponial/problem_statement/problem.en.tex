\problemname{Exponial}
\newcommand{\exponial}{\operatorname{exponial}}
\illustration{0.30}{9_West_57th_Street}{Illustration of $\exponial(3)$ (not to scale), \href{https://en.wikipedia.org/wiki/File:Manhattan_-_9_West_57th_Street.JPG}{Picture} by C.M. de Talleyrand-Périgord via Wikimedia Commons}%
\noindent
Everybody loves big numbers (if you do not, you might want to stop
reading at this point).  There are many ways of constructing really
big numbers known to humankind, for instance:
\begin{itemize}
\item Exponentiation: $42^{2016} = \underbrace{42 \cdot 42 \cdot \ldots \cdot 42}_{2016\text{ times}}$.
\item Factorials: $2016! = 2016 \cdot 2015 \cdot
\ldots \cdot 2 \cdot 1$.
\end{itemize}

In this problem we look at their lesser-known love-child the
\emph{exponial}, which is an operation defined for all positive
integers $n$ as
\[
\exponial(n) = n^{(n-1)^{(n-2)^{\cdots^{2^{1}}}}}.
\]
For example, $\exponial(1) = 1$ and $\exponial(5) =
5^{4^{3^{2^1}}} \approx 6.206 \cdot 10^{183230}$ which is already pretty
big.  Note that exponentiation is right-associative: $a^{b^c} = a^{(b^c)}$.

Since the exponials are really big, they can be a bit unwieldy to work
with.  Therefore we would like you to write a program which computes
$\exponial(n) \bmod m$ (the remainder of $\exponial(n)$ when dividing
by $m$).

\section*{Input}

The input consists of two integers $n$ ($1 \le n \le 10^9$) and $m$ ($1 \le m \le 10^9$).

\section*{Output}

Output a single integer, the value of $\exponial(n) \bmod m$.
