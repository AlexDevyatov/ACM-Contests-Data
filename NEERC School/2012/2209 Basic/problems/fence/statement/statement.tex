\begin{problem}{Забор}{fence.in}{fence.out}{2 секунды}

%Автор задачи: Владимир Ульянцев
%Автор условия: Павел Кротков

Забор вокруг больницы Принстон-Плейнсборо был построен задолго до появления там доктора Хауса, доктора Кадди и всех остальных ее сотрудников, 
ныне там работающих. Забор представляет собой выпуклый многоугольник. В вершинах этого многоугольника стоят столбы, а его ребрами являются секции 
забора. Однако, при строительстве забора по ошибке был построен лишний столб, который в итоге оказался внутри забора, то есть на территории больницы, 
и причинял персоналу и больным много неудобств.

Доктор Кадди решила исправить это досадное упущение, построив вокруг больницы новый забор. Новый забор должен удовлетворять трем условиям:

\begin{itemize}
\item вершинами многоугольника нового забора могут быть только уже существующие столбы (вершины старого и данный столб внутри него)
\item внутри многоугольника, представляющего новый забор, не должно быть столбов, не являющихся вершинами многоугольника
\item его площадь должна быть максимально возможной
\end{itemize}

При этом многоугольник, представляющий новый забор, может быть невыпуклым, а также некоторые столбы, находящиеся за его территорией, могут остаться 
неиспользованными.

\InputFile
Первая строка входного файла содержит одно целое число $n$ ($3 \le n \le 10^5$)~--- количество вершин в многоугольнике, представляющем старый забор. 
Следующие $n$ строк содержат по два целых числа $x$ и $y$, не превосходящих по абсолютной величине $10^8$~--- координаты вершин этого многоугольника. 
Координаты даны в порядке обхода многоугольника против часовой стрелки, многоугольник выпуклый. Никакие три вершины многоугольника не лежат на одной прямой.

Последняя строка входного файла содержит два целых числа $X$ и $Y$, не превосходящих по абсолютной величине $10^8$~--- координаты лишнего столба. 
Гарантируется, что точка ($X$, $Y$) лежит строго внутри многоугольника.

\OutputFile               
Опишите многоугольник, представляющий новый забор, в том же формате, в котором описан старый. Внутри нового забора не должно быть столбов и его 
площадь должна быть максимальна.

\Examples
\begin{example}%
\exmp{
4
-1 -1
1 -1
1 1
-1 1
0 0
}{
5
-1 -1
0 0
-1 1
1 1
1 -1
}%
\end{example}
\end{problem}
