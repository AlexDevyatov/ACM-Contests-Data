\begin{problem}{Госпиталь}{hospital.in}{hospital.out}{2 секунды}

%Автор задачи: Николай Ведерников
%Автор условия: Андрей Комаров

В городе объявлена эпидемия волчанки. Имеющиеся госпитали не справляются
с наплывом больных. Администрацией было решено вызвать эксперта по 
борьбе с волчанкой~--- Грегори Хауса. Но этот мизантроп отказывается 
работать в команде с кем-либо в имеющихся госпиталях и требует себе 
новый. Город надо спасать, поэтому его требование решено было удовлетворить
и новый госпиталь построить. Теперь необходимо выбрать место, на
котором он будет построен.

Город представляет из себя $n$ площадей, некоторые из которых 
соеденены дорогами. Причём, от любой площади до любой другой можно
доехать единственным способом. На $i$-й площади живёт $a_i$ людей. 
После открытия госпиталя Хауса, наслушанные рассказами о его профессионализме,
все люди пойдут в день открытия в этот госпиталь, чтобы попасть к Хаусу
на осмотр. Это учитывается при выборе места строительства, и для 
подхода с каждой стороны будет своя дверь. К каждой двери выстроится 
своя очередь людей, которые подошли с этой стороны. Администрации
госпиталя не хочется, чтобы людям показалось, что будут огромные очереди,
и поэтому, они хотят минимизировать длину самой длинной очереди 
ко входу. Помогите выбрать такое место для госпиталя, чтобы самая длинная
очередь была как можно короче.

\InputFile
В первой строке задано число площадей $n$ ($1 \le n \le 100{\,}000$).
Во второй строке заданы $n$ чисел $a_i$ ($1 \le a_i \le 10^9$)~--- 
населённости площадей.
Далее, в $n-1$-й строке, заданы номера соединённых дорогой площадей.

\OutputFile               
Выведите единственное число~--- номер площади, на которой можно 
построить госпиталь. Если ответов несколько, выведите любой.

\Examples
\begin{example}%
\exmp{
5
3 3 2 5 1
1 2
2 3
2 4
4 5
}{
2
}%
\end{example}

\end{problem}
