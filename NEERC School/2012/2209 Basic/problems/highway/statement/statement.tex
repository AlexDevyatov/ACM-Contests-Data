\begin{problem}{Шоссе}{highway.in}{highway.out}{2 секунды}

%Автор задачи: Алексей Цыпленков
%Автор условия: Нияз Нигматуллин

%Есть прямоугольник HxW, по которому двигаются слева направо параллельно Ox
%прямоугольники поменьше. Все прямоугольники двигаются с одной скоростью. Проекции
%никаких двух прямоугольников на Oy не пересекаются. Когда прямоугольник доезжает до
%правого края, он нацинает выползать из-за левого.

%Есть чувачок, который хочет перейти прямоугольник снизу вверх. Он умеет двигаться
%с некоторой известной постоянной скоростью v, останавливаться не умеет. Сказать,
%может ли он перейти.

Доктор Реми Хадли, более известная как Тринадцатая, больна хореей Хантингтона. При этой болезни в какой-то момент с человеческим разумом начинают 
происходить необратимые изменения. В том числе, резко снижаются интеллектуальные способности. Чтобы ни в коем случае не упустить этот момент и 
вовремя начать агрессивную терапию, Тринадцатая каждую неделю выполняет несложное упражнение, заключающееся в прохождении компьютерной игры.

В игре предлагается перейти шоссе, по которому двигаются автомобили. Шоссе представляет из себя прямоугольник размера $W{\times}H$ метров, а 
автомобили~--- прямоугольники меньшего размера, расположенные внутри него. В левом нижнем углу, в точке с координатами (0, 0), расположен человек.

\begin{center}
	\includegraphics{../problems/highway/statement/pics/pic.1}\\
	Состояние шоссе в момент времени $t = 0$.
\end{center}

Каждый автомобиль непрерывно двигается по шоссе вправо со скоростью один метр в секунду. При этом, как только автомобиль касается правого края шоссе, 
он начинает исчезать справа и появляться слева с той же скоростью.

\begin{center}
	\includegraphics{../problems/highway/statement/pics/pic.2}\\
	Состояние шоссе в момент времени $t = 2$.
\end{center}
 
Человеку необходимо перейти шоссе. Как только он решает это сделать, он начинает непрерывно двигаться вертикально вверх со скоростью один метр в 
секунду. При этом, возможность остановиться у него отсутствует. Если в какой то момент времени он оказывается строго внутри какого-то автомобиля, 
он умирает. Если же до момента достижения верхней границы шоссе он не касается автомобилей или попадает на их границы, то он успешно переходит 
шоссе.

Задача доктора Хадли состоит в определении того, может ли человек успешно перейти шоссе. Кроме того, если у него есть эта возможность, необходимо 
определить количество секунд, через которое он должен начать движение.

\InputFile
В первой строке входного файла заданы два целых числа: $W$ и $H$ ($4 \le W, H \le 10^4$)~--- длина и ширина шоссе соответственно.

Во второй строке задано целое число $n$ ($1 \le n \le 100{\,}000$)~--- количество автомобилей в начальный момент времени.

В следующих $n$ строках заданы автомобили, по одному в строке четырьмя целыми числами: $x_1$, $y_1$, $x_2$, $y_2$~--- 
координаты противоположных углов соответствующего автомобилю прямоугольника ($0 \le x_1, x_2 \le W$, $0 \le y_1, y_2 \le H$, $x_1 \neq x_2$, 
$y_1 \neq y_2$).

Гарантируется, что прямоугольники, соответствующие автомобилям, не пересекаются и не касаются друг друга.

\OutputFile               

Если человек может успешно перейти шоссе, в первой строке выходного файла выведите \texttt{<<Yes>>}. Во второй строке выведите одно вещественное число 
$t$ ($0 \le t \le W$) ~--- время в секундах, через которое он может начинать движение. Ответ выводите с максимально возможной точностью.

В противном случае выведите в выходной файл \texttt{<<No>>}.

\Examples
\begin{example}%
\exmp{
8 4
2
0 0 3 2
5 4 7 1
}{
Yes
3.0
}%
\end{example}
\end{problem}
