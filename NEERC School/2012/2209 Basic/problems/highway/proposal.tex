\begin{proposal}{Странное шоссе}{highway.in}{highway.out}{}

\Author                  
Алексей Цыпленков

\ProblemIdea
Есть прямоугольник HxW, по которому двигаются слева направо параллельно Ox
прямоугольники поменьше. Все прямоугольники двигаются с одной скоростью. Проекции
никаких двух прямоугольников на Oy не пересекаются. Когда прямоугольник доезжает до
правого края, он нацинает выползать из-за левого.

Есть чувачок, который хочет перейти прямоугольник снизу вверх. Он умеет двигаться
с некоторой известной постоянной скоростью v, останавливаться не умеет. Сказать,
может ли он перейти.
                    
\ProblemVariations
Куча произвольных усложнейний, как то:
\begin{itemize}
\item Две скорости движения.
\item N скоростей движения.
\item Два направления движения(здесь правда нужно еще подумать над решением).
\end{itemize}

\SolutionIdea
Сдвинем каждую точку границы каждого прямоугольника на v * (y / v), y - соответствующая 
координата точки. В результате прямоугольники превратятся в параллелограммы и нужно
будет найти вертикальную прямую, которая не пересекается ни с каким параллелограммом.
Касаться можно.

\ProblemComplexity
Идея~--- средняя (M). Реализация~--- средняя (M).

\end{proposal}
